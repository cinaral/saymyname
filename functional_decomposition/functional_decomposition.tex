\documentclass[tikz,preview]{standalone} %
\usepackage{tikz}
\usetikzlibrary{shapes,arrows,positioning,calc}
\usepackage{tikz}
\usetikzlibrary{matrix,calc,shapes,arrows}

\begin{document}
% Define block styles
\centering
\tikzstyle{decision} = [diamond, draw, fill=blue!20, 
text width=4.5em, text badly centered, node distance=3cm, inner sep=0pt]
\tikzstyle{empty} = [rectangle, 
text width=10em, text centered, rounded corners, minimum height=4em]
\tikzstyle{block} = [rectangle, draw, fill=blue!20, 
text width=8em, text centered, rounded corners, minimum height=4em]
\tikzstyle{line} = [draw, -latex']
\tikzstyle{cloud} = [draw, ellipse,fill=red!20, node distance=3cm,text width=7em, text centered,
minimum height=3em]
\pagestyle{empty}

\begin{tikzpicture}[node distance = 2cm, auto]
	% Place nodes
	\node [block] (header) {RoboFish};
	\node [cloud, below of=header, node distance=2 cm] (sub13) {Communicate with the fish};
	\node [cloud, left of=sub13, node distance= 5 cm] (sub12) {Communicate with the environment};
	\node [cloud, left of=sub12, node distance=5 cm] (sub11) {Move the aquarium};
	\node [cloud, right of=sub13, node distance=5 cm] (sub14) {Contain the water and fish};
	\node [cloud, right of=sub14, node distance=5 cm] (sub15) {Power the robot};
	\node [empty, left of=sub11, node distance=2 cm] (subsub10) {};
	\node [empty, right of=subsub10, node distance=4 cm] (subsub100) {};
	\node [block,  below of=subsub10] (subsub11) {Carry the aquarium};
	\node [block, below of=subsub11] (subsub12) {Detect the position of the fish};
	\node [block, below of=subsub12] (subsub13) {Detect obstacles};
	\node [block, below of=subsub13] (subsub14) {Detect the position of robot};
	\node [block, below of=subsub100, node distance=3cm] (subsub15) {Faciliate the movement};
	\node [block, below of=subsub15] (subsub16) {Actuate the movement};
	\node [block, below of=subsub16] (subsub17) {Control the movement};
	\node [empty, right of=sub12, node distance= 2 cm] (subsub20) {};
	\node [block, below of=subsub20] (subsub21) {Send and receive messages over the network};
	\node [block, below of=subsub21] (subsub22) {Display the current status to nearby observers};
	\node [empty, left of=sub15, node distance= 2 cm] (subsub50) {};
	\node [block, below of=subsub50] (subsub51) {Store the power};
	\node [block, below of=subsub51] (subsub52) {Charge the storage};    
	% Draw edges
	\path [line] (header) -|  node[near end] {1} (sub11);
	\path [line] (header) -|  node[near end] {2} (sub12);
	\path [line] (header) -- node {3} (sub13);
	\path [line] (header) -| node[near end] {4} (sub14);
	\path [line] (header) -|  node[near end] {5} (sub15);
	\path [line] (sub11) |- node[near end] {1-1} (subsub11);
	\path [line] (sub11) |- node[near end] {1-2} (subsub12);
	\path [line] (sub11) |- node[near end] {1-3} (subsub13);
	\path [line] (sub11) |- node[near end] {1-4} (subsub14);
	\path [line] (sub11) |- node[near end] {1-5} (subsub15);
	\path [line] (sub11) |- node[near end] {1-6} (subsub16);
	\path [line] (sub11) |- node[near end] {1-7} (subsub17);
	\path [line] (sub12) |- node[near end] {2-1} (subsub21);
	\path [line] (sub12) |- node[near end] {2-2} (subsub22);
	\path [line] (sub15) |- node[near end] {5-1} (subsub51);
	\path [line] (sub15) |- node[near end] {5-2} (subsub52);

\end{tikzpicture}
\end{document}