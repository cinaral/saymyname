\documentclass[a4paper, 10pt, DIV=16, parskip = full, twocolumn = true]{scrartcl}
\usepackage{setspace}
\usepackage{amsmath}
\usepackage{amssymb}
\usepackage{lastpage}
\usepackage{scrlayer-scrpage}
\usepackage[utf8]{inputenc}
\usepackage{graphicx}
%\usepackage{subcaption}
\usepackage[section]{placeins}
%\usepackage{pdfpages}
%\usepackage{layout}
\usepackage{enumitem}
\usepackage{booktabs}

\usepackage{placeins}
\usepackage{inconsolata}
\usepackage[bottom]{footmisc}
\usepackage[group-digits=false]{siunitx}
\usepackage{caption}

%\usepackage{lipsum}
\usepackage{cuted}

%\setlength{\parskip}{3pt}
\setlength{\columnsep}{15pt}
%\onehalfspacing
%\setlength{\columnseprule}{0.1pt}
\cofoot{\thepage\ of \pageref{LastPage}}
\subject{METU ME462 Mechatronic Design, Spring 2020}
\title{Overall Design Evaluation: Design and manufacture of a wheeled robot controlled by the movement of a fish swimming inside an aquarium}
\author{\textbf{Say My Name} \\ Sercan Aslan / 1909902 \\ Ali Levent Çınar / 2234532 \\ Yusuf Can Coşkun / 1939420}
\date{March 9, 2020}

\renewcommand{\familydefault}{\sfdefault}
\renewcommand*\thesection{\arabic{section}.}
%\renewcommand*\thesubsection{\!\!\!}
\renewcommand{\theequation}{\arabic{equation}}
\renewcommand{\thefigure}{\arabic{figure}}
\renewcommand{\thetable}{\arabic{table}}

% squeeze tables
\renewcommand{\arraystretch}{0.7}
\renewcommand{\tabcolsep}{1mm}

\begin{document}
	\maketitle
	\thispagestyle{scrheadings}
	%\includepdf[pages=-]{Images/Title-Page.pdf} % set twocolumn = false
	\pagenumbering{arabic}

%%%%%%%%%%%%%%%%%%%%%%%%%%%%%%%%%%%%%%%%%%%%%%%%%%%%%%%%%%%%%%%%%%%%%%%%%%%%%%%%%%%%%%%%%%%%%%%%%%%%%%%%%%%	
%\section{Introduction}

\begin{strip}
	Design alternatives has been explored for each function of a wheeled robot controlled by the movement of a fish swimming inside an aquarium (Robofish). The generated overall concepts and their respective evaluation criteria have been summarized in Table \ref{table:concepts} and \ref{table:criterion}. Note that the concepts and evaluation criteria will be referred by their concept ID (OCID) and evaluation ID (OEID), respectively, in the remainder of this document. The CID of the highest scoring concepts will be presented in bold throughout the document. Datum concepts will be denoted with a star. The evaluation criteria are weighted using the analytic hierarchy process (AHP) in Tables \ref{table:AHP}. The generated concepts are systematically evaluated in Tables \ref{table:pugh} using weighted decision-matrix (Pugh) method.
\end{strip}

%%%%%%%%%%%%%%%%%%%%%%%%%%%%%%%%%%%%%%%%%%%%%%%%%%%%%%%%%%%%%%%%%%%%%%%%%%%%%%%%%%%%%%%%%%%%%%%%

\begin{table*}
\centering
\caption{Generated overall concepts for the functions of Robofish (important characteristics in bold).}
	\begin{tabular}{rl}
	\toprule
		OCID & Overall Concept \\
	\midrule
		OC1 & \textbf{Hexagonal} base robot with three \textbf{omni} wheels. \\ 
		& The fish position is detected by a webcam placed \textbf{below} the hexagonal aquarium. \\
		& The robot position is detected using \textbf{three cameras} facing in different directions. \\
		& Includes three DC motors, Raspberry Pie, ultrasonic sensors, LED indicator lights, top cover. \\
		& Powered by a dry accumulator and recharged through its dangling magnet receptor.\\
		OC2 & \textbf{Hexagonal} base robot with three \textbf{omni} wheels. \\
		& The fish position is detected by a webcam placed \textbf{above} the hexagonal aquarium. \\
		& The robot position is detected using a \textbf{wide lens camera} place above the aquarium. \\
		& Includes three DC motors, Raspberry Pie, ultrasonic sensors, LED indicator lights, top cover. \\
		& It's powered by a dry accumulator and recharged through its dangling magnet receptor.\\
		OC3 & \textbf{Square} base robot with four \textbf{mecanum} wheels. \\
		& Aquarium is supported by a \textbf{basket} shaped base. \\
		& The fish position is detected by a webcam placed \textbf{above} the aquarium using supporting beams. \\
		& The robot position is detected using a \textbf{wide lens camera} place above the aquarium. \\ 
		& Includes three DC motors, Raspberry Pie, ultrasonic sensors, LED indicator lights, top cover. \\
		& It's powered by a dry accumulator and recharged through its dangling magnet receptor.\\
		OC4 & \textbf{Square} base robot with four \textbf{mecanum} wheels. \\
		& Aquarium is supported by a supporting rack, can be removed from the \textbf{rack} with a sliding motion. \\
		& The fish position is detected by a webcam placed \textbf{above} the aquarium using supporting beams. \\
		& The robot position is detected using a \textbf{wide lens camera} place above the aquarium. \\
		& Includes three DC motors, Raspberry Pie, ultrasonic sensors, LED indicator lights, top cover. \\
		& It's powered by a dry accumulator and recharged through its dangling magnet receptor.\\
	\bottomrule	
	\end{tabular}
\label{table:concepts}
\end{table*}

\begin{table*}
	\centering
	\caption{Concept evaluation criteria for the overall concepts of Robofish}
	\begin{tabular}{rl}
		\toprule
		OEID & Overall Evaluation Criterion \\
		\midrule
		OE1 & Mechanical stability \\
		OE2 & Aesthetics \\	
		OE3 & Cost \\	
		OE4 & Aquarium removability \\		
		OE5 & Ease of manufacturing \\	
		OE6 & Holonomic motion ability \\
		OE7 & Weight carrying ability\\	
		\bottomrule
	\end{tabular}
\label{table:criterion}
\end{table*}

%%%%%%%%%%%%%%%%%%%%%%%%%%%%%%%%%%%%%%%%%%%%%%%%%%%%%%%%%%%%%%%%%%%%%%%%%%%%%%%%%%%%%%%%%%%%%%%%
%\section{Criteria weighting}

\begin{table*}
	\centering
	\caption{AHP for concept evaluation criteria for the overall concepts.}
	\begin{tabular}{l|sssssssS}
		\toprule
		& \text{OE1} & \text{OE2} & \text{OE3} & \text{OE4} & \text{OE5} & \text{OE6} & \text{OE7} & \text{Weight (\%)} \\
		\midrule
		OE1 & 1 & 5 & 3 & 3 & 3 & 3 & 1 & 20.3 \\
		OE2 & & 1 & 3 & 1/3 & 3 & 1/5 & 1/5 & 7.45 \\
		OE3 & & & 1 & 1/5 & 1/3 & 1/5 & 1/3 & 4.71 \\
		OE4 & & & & 1 & 3 & 1/3 & 3 & 15.8 \\
		OE5 & & & & & 1 & 1/3 & 1/3 & 8.56 \\
		OE6 & & & & & & 1 & 5 & 25.7 \\
		OE7 & & & & & & & 1 & 17.5 \\
	\bottomrule
	\end{tabular}
	\label{table:AHP}
	
	\centering
	\caption{Decision-matrix for the overall concepts.}
	\begin{tabular}{rS|cccc}
		\toprule
		& \text{(\%)} & \textbf{OC1} & \text{OC2} & \text{OC3*} & \text{OC4} \\
		\midrule
		OE1 & 20.3 & 1 & 1 & D & 0 \\
		OE2 & 7.45 & 1 & 1 & D & 1 \\
		OE3 & 4.71 & 1 & 1 & D & 0 \\
		OE4 & 15.8 & 0 & -1 & D & 1 \\
		OE5 & 8.56 & -1 & -1 & D & -1 \\
		OE6 & 25.7 & 1 & 1 & D & 0 \\
		OE7 & 17.5 & -1 & -1 & D & 1 \\
		\midrule
		& Total & \textbf{0.322} & \text{0.163} & 0 & \text{0.321} \\
		\bottomrule
	\end{tabular}
	\label{table:pugh}
\end{table*}

\end{document}